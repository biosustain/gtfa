\documentclass[11pt]{article}

\usepackage{sectsty}
\usepackage{graphicx}
\usepackage[english]{babel}
\usepackage{csquotes}
\usepackage[backend=biber]{biblatex}
\usepackage{todonotes} % For taking notes while writing

\usepackage{amsmath}


% Margins
\topmargin=-0.45in
\evensidemargin=0in
\oddsidemargin=0in
\textwidth=6.5in
\textheight=9.0in
\headsep=0.25in

\title{ Title}
\author{ Author }
\date{\today}


\addbibresource{references.bib}

\begin{document}
	\maketitle	
	\pagebreak
	
	% Optional TOC
	% \tableofcontents
	% \pagebreak

\section{Abstract}

\section{Introduction}
% Introduction to metabolic modelling

% Thermo

% Inconsistency - look at preprint

% Problems with TMFA modelling approach

% Contraints need to be relaxed


Both \textcite{gollub_2021_prob_sampling} and \textcite{vishnu_2021_multiTFA} addressed the problem of excessively permissive univariate constraints on metabolite concentrations by reformulating them as a 95\% confidence ellipsoid. This was shown to massively reduce the number of viable direction configurations \cite{gollub_2021_prob_sampling}. \textcite{gollub_2021_prob_sampling} further developed methods for sampling reaction Gibbs energies and fluxes with a Hit and Run Markov Chain Monte Carlo (MCMC) approach. This was necessary to address the disconnected and discontinuous parameter space, which is fragmented by orthants (i.e. directionality configurations) that do not allow any flux. Their approach on \textit{E. coli} model iML1515 \cite{monk_2017_ecoli} suggests that uncertainty in $\Delta_rG^\deg$ is of less importance than uncertainties in metabolite concentrations $c$.  \todo{Will need actual symbols here}.  
\section{Methods}	
\subsection{A generative statistical model implementing thermodynamic flux analysis}

This study proposes a new method for incorporating thermodynamic information
into an analysis of a metabolic network. We create a joint generative
statistical model of measurements of four kinds of quantity: metabolic fluxes,
metabolite concentrations, enzyme concentrations and Gibbs free energies of
reaction. These are treated as interconnected, allowing for more precise
estimates than would be possible with independent analyses.

\subsection{Generative model}

Fluxes are treated as determined by Gibbs free energies of reaction, enzyme
concentrations and a latent parameter vector $b$, which can be interpreted as
the amount of flux carried by each enzyme at steady state. The flux $v_{ij}$ of
reaction $i$ in condition {j} is as follows:

$$
v_{ij} = \Delta_rG_i \cdot e_{ij} \cdot b_{ij}
$$

Gibbs free energies of reaction are treated as determined by metabolite
concentrations and formation energies as follows:

$$
\Delta_rG' = S^T(\Delta_fG' + RT\ln c)
$$

Standard condition measurements of reaction gibbs free energies (as can be
derived, for example, from the TECRDB database) and metabolic fluxes (as
derived from fluxomics analysis) are represented using a standard linear
regression model:

\begin{align*}
	y_{\Delta_rG} &\sim N(\Delta_rG, \sigma_{\Delta_rG}) \\
	y_{v} &\sim N(v, \sigma_{v})
\end{align*}

Measurements of metabolite and enzyme concentrations, as derived from
metabolomics and proteomics analyses, are represented using a lognormal
generalised linear model:

\begin{align*}
	y_{c} & \sim LN(\ln(c), \sigma_{c}) \\
	y_{e} & \sim LN(\ln(e), \sigma_{e})
\end{align*}

This model is generative in the sense that, given an assignment of values to
the unknown parameters $\Delta_fG$, $c$, $e$, $b$, $\sigma_{\Delta_rG}$,
$\sigma_{v}$, $\sigma_c$ and $\sigma_{e}$ it is possible to simulate new values
for the measured quantities $y_{\Delta_rG}$, $y_v$, $y_c$ and $y_e$. The model
therefore represents a theory as to how the observed data was generated. The
theory can be tested both by comparing its predictions with real data and by
assessing the plausibility of its parameters.

\subsection{Contrast with traditional tfa}

Traditional thermodynamic flux analysis (TFA) seeks to improve analyses of
metabolic networks by taking advantage of information about the thermodynamic
properties of the chemical reactions involved. TFA has historically been
carried out within a constraint-based framework according to which the flux
profile of a biological system is predicted by optimising an objective function
representing the system's goals, subject to constraints imposed by the
available information. For example, according to the mass balance constraint,
metabolic fluxes must leave the system in a steady state.

In this framework thermodynamic information allows extra constraints to be
imposed, representing the fact that the amount and direction of the flux a
reaction carries is partly determined by its thermodynamic properties.

\section{Results and discussion}


\section{Conclusion}

\printbibliography

\end{document}

