%! Author = jason
%! Date = 27/10/21

% Template for PLoS
% Version 3.5 March 2018
%
% % % % % % % % % % % % % % % % % % % % % %
%
% -- IMPORTANT NOTE
%
% This template contains comments intended
% to minimize problems and delays during our production
% process. Please follow the template instructions
% whenever possible.
%
% % % % % % % % % % % % % % % % % % % % % % %
%
% Once your paper is accepted for publication,
% PLEASE REMOVE ALL TRACKED CHANGES in this file
% and leave only the final text of your manuscript.
% PLOS recommends the use of latexdiff to track changes during review, as this will help to maintain a clean tex file.
% Visit https://www.ctan.org/pkg/latexdiff?lang=en for info or contact us at latex@plos.org.
%
%
% There are no restrictions on package use within the LaTeX files except that
% no packages listed in the template may be deleted.
%
% Please do not include colors or graphics in the text.
%
% The manuscript LaTeX source should be contained within a single file (do not use \input, \externaldocument, or similar commands).
%
% % % % % % % % % % % % % % % % % % % % % % %
%
% -- FIGURES AND TABLES
%
% Please include tables/figure captions directly after the paragraph where they are first cited in the text.
%
% DO NOT INCLUDE GRAPHICS IN YOUR MANUSCRIPT
% - Figures should be uploaded separately from your manuscript file.
% - Figures generated using LaTeX should be extracted and removed from the PDF before submission.
% - Figures containing multiple panels/subfigures must be combined into one image file before submission.
% For figure citations, please use "Fig" instead of "Figure".
% See http://journals.plos.org/plosone/s/figures for PLOS figure guidelines.
%
% Tables should be cell-based and may not contain:
% - spacing/line breaks within cells to alter layout or alignment
% - do not nest tabular environments (no tabular environments within tabular environments)
% - no graphics or colored text (cell background color/shading OK)
% See http://journals.plos.org/plosone/s/tables for table guidelines.
%
% For tables that exceed the width of the text column, use the adjustwidth environment as illustrated in the example table in text below.
%
% % % % % % % % % % % % % % % % % % % % % % % %
%
% -- EQUATIONS, MATH SYMBOLS, SUBSCRIPTS, AND SUPERSCRIPTS
%
% IMPORTANT
% Below are a few tips to help format your equations and other special characters according to our specifications. For more tips to help reduce the possibility of formatting errors during conversion, please see our LaTeX guidelines at http://journals.plos.org/plosone/s/latex
%
% For inline equations, please be sure to include all portions of an equation in the math environment.  For example, x$^2$ is incorrect; this should be formatted as $x^2$ (or $\mathrm{x}^2$ if the romanized font is desired).
%
% Do not include text that is not math in the math environment. For example, CO2 should be written as CO\textsubscript{2} instead of CO$_2$.
%
% Please add line breaks to long display equations when possible in order to fit size of the column.
%
% For inline equations, please do not include punctuation (commas, etc) within the math environment unless this is part of the equation.
%
% When adding superscript or subscripts outside of brackets/braces, please group using {}.  For example, change "[U(D,E,\gamma)]^2" to "{[U(D,E,\gamma)]}^2".
%
% Do not use \cal for caligraphic font.  Instead, use \mathcal{}
%
% % % % % % % % % % % % % % % % % % % % % % % %
%
% Please contact latex@plos.org with any questions.
%
% % % % % % % % % % % % % % % % % % % % % % % %

\documentclass[10pt,letterpaper]{article}
\usepackage[top=0.85in,left=2.75in,footskip=0.75in]{geometry}

% amsmath and amssymb packages, useful for mathematical formulas and symbols
\usepackage{amsmath,amssymb}

% Use adjustwidth environment to exceed column width (see example table in text)
\usepackage{changepage}

% Use Unicode characters when possible
\usepackage[utf8x]{inputenc}

% textcomp package and marvosym package for additional characters
\usepackage{textcomp,marvosym}
\usepackage{gensymb}

% cite package, to clean up citations in the main text. Do not remove.
\usepackage{cite}

% Use nameref to cite supporting information files (see Supporting Information section for more info)
\usepackage{nameref,hyperref}

% line numbers
\usepackage[right]{lineno}

% ligatures disabled
\usepackage{microtype}
\DisableLigatures[f]{encoding = *, family = * }

% color can be used to apply background shading to table cells only
\usepackage[table]{xcolor}

% array package and thick rules for tables
\usepackage{array}

% create "+" rule type for thick vertical lines
\newcolumntype{+}{!{\vrule width 2pt}}

% create \thickcline for thick horizontal lines of variable length
\newlength\savedwidth
\newcommand\thickcline[1]{%
  \noalign{\global\savedwidth\arrayrulewidth\global\arrayrulewidth 2pt}%
  \cline{#1}%
  \noalign{\vskip\arrayrulewidth}%
  \noalign{\global\arrayrulewidth\savedwidth}%
}

% \thickhline command for thick horizontal lines that span the table
\newcommand\thickhline{\noalign{\global\savedwidth\arrayrulewidth\global\arrayrulewidth 2pt}%
\hline
\noalign{\global\arrayrulewidth\savedwidth}}


% Remove comment for double spacing
%\usepackage{setspace}
%\doublespacing

% Text layout
\raggedright
\setlength{\parindent}{0.5cm}
\textwidth 5.25in
\textheight 8.75in

% Bold the 'Figure #' in the caption and separate it from the title/caption with a period
% Captions will be left justified
\usepackage[aboveskip=1pt,labelfont=bf,labelsep=period,justification=raggedright,singlelinecheck=off]{caption}
\renewcommand{\figurename}{Fig}

% Use the PLoS provided BiBTeX style
\bibliographystyle{plos2015}

% Remove brackets from numbering in List of References
\makeatletter
\renewcommand{\@biblabel}[1]{\quad#1.}
\makeatother



% Header and Footer with logo
\usepackage{lastpage,fancyhdr,graphicx}
\usepackage{epstopdf}
\usepackage{amsfonts}
\usepackage{natbib}
%\pagestyle{myheadings}
\pagestyle{fancy}
\fancyhf{}
%\setlength{\headheight}{27.023pt}
%\lhead{\includegraphics[width=2.0in]{PLOS-submission.eps}}
\rfoot{\thepage/\pageref{LastPage}}
\renewcommand{\headrulewidth}{0pt}
\renewcommand{\footrule}{\hrule height 2pt \vspace{2mm}}
\fancyheadoffset[L]{2.25in}
\fancyfootoffset[L]{2.25in}
\lfoot{\today}

%% Include all macros below

%\newcommand{\lorem}{{\bf LOREM}}
%\newcommand{\ipsum}{{\bf IPSUM}}


\newcommand{\dgf}{\Delta_fG}
\newcommand{\sdgf}{\Delta_fG^{\degree}}
\newcommand{\dgr}{\Delta_rG}
\newcommand{\sdgr}{\Delta_rG^{\degree}}
\newcommand{\sdgg}{\Delta_gG}
\newcommand{\bsdgg}{\mathbf{\sdgg}}
\newcommand{\bdgf}{\mathbf{\dgf}}
\newcommand{\bsdgf}{\mathbf{\sdgf}}
\newcommand{\bdgr}{\mathbf{\dgr}}
\newcommand{\bsdgr}{\mathbf{\sdgr}}
\newcommand{\be}{\mathbf{e}}
\newcommand{\bc}{\mathbf{c}}
\newcommand{\bb}{\mathbf{b}}
\newcommand{\bv}{\mathbf{v}}
\newcommand{\bbe}{\mathbf{be}}
% Projection matrices
\newcommand{\projr}{\mathbf{P}_{\mathit{R}(S)}}
\newcommand{\projn}{\mathbf{P}_{\mathit{N}(S^T)}}




%% END MACROS SECTION


\begin{document}

\vspace*{0.2in}

% Title must be 250 characters or less.
\begin{flushleft}
{\Large
\textbf\newline{GTFA: Genrative Thermodynamic Flux Analysis} % Please use "sentence case" for title and headings (capitalize only the first word in a title (or heading), the first word in a subtitle (or subheading), and any proper nouns).
}
\newline
% Insert author names, affiliations and corresponding author email (do not include titles, positions, or degrees).
\\
Jason Jooste\textsuperscript{1},
Teddy Groves\textsuperscript{2},
Lars Nielsen\textsuperscript{2,3},
%Name4 Surname\textsuperscript{2},
%Name5 Surname\textsuperscript{2\ddag},
%Name6 Surname\textsuperscript{2\ddag},
%Name7 Surname\textsuperscript{1,2,3*},
%with the Lorem Ipsum Consortium\textsuperscript{\textpilcrow}
\\
\bigskip
\textbf{1} Ludwig Maximlian's Universität München, Munich, Bavaria, Germany
\\
\textbf{2} Novo Nordisk Foundation Center for Biosustainability, Technical University of Denmark, Kgs. Lyngby, Denmark
%\\
\textbf{3} Australian Institute for Bioengineering and Nanotechnology, University of Queensland, Brisbane, Queensland, Australia
%\\
\bigskip

% Insert additional author notes using the symbols described below. Insert symbol callouts after author names as necessary.
%
% Remove or comment out the author notes below if they aren't used.
%
% Primary Equal Contribution Note
%\Yinyang These authors contributed equally to this work.

% Additional Equal Contribution Note
% Also use this double-dagger symbol for special authorship notes, such as senior authorship.
%\ddag These authors also contributed equally to this work.

% Current address notes
%\textcurrency Current Address: Dept/Program/Center, Institution Name, City, State, Country % change symbol to "\textcurrency a" if more than one current address note
% \textcurrency b Insert second current address
% \textcurrency c Insert third current address

% Deceased author note
%\dag Deceased

% Group/Consortium Author Note
%\textpilcrow Membership list can be found in the Acknowledgments section.

% Use the asterisk to denote corresponding authorship and provide email address in note below.
%* correspondingauthor@institute.edu

\end{flushleft}
% Please keep the abstract below 300 words
\section*{Abstract}


% Please keep the Author Summary between 150 and 200 words
% Use first person. PLOS ONE authors please skip this step.
% Author Summary not valid for PLOS ONE submissions.
\section*{Author summary}

%Aim to highlight where your work fits within a broader context; present the significance or possible implications of
%your work simply and objectively; and avoid the use of acronyms and complex terminology wherever possible. The goal is
%to make your findings accessible to a wide audience that includes both scientists and non-scientists.

\linenumbers

% Use "Eq" instead of "Equation" for equation citations.
\section*{Introduction}

% TODO: Longer intro to the field in general?
Modelling of feasible intracellular fluxes has long been a key task in metabolic engineering, with the end goal being the most accurate picture of internal fluxes of a microorganism given sparse data.
If little data is available, the space of valid fluxes can be relatively large, limited only by fundamental reaction stoichiometry and mass-balance laws
\[
    \label{eq:steady_state}
    0 = S\cdot \mathbf{v}
\]

% QUOTE FROM PAPER: Include A positive stoichiometric coefficient means that the metabolite is produced by reaction An and a negative entry indicates that the metabolite is consumed in that reaction.

with $S \in \mathbb{Z}^{m\times r}$ representing the stoichiometric matrix, which contains the stoichiometric coefficients of the $m$ metabolites in the $r$ reactions in the metabolic network and $\mathbf{v} \in \mathbb{R}^r$ representing the fluxes of reactions $1,\dots,r$.
If more data is available, this space can be further constrained by known thermodynamics, kinetics and regulatory interactions of metabolic enzymes.
From this space of valid fluxes a single most-parsimonious solution can then be obtained, with parsiomny determined by maximising biomass production, nearness to measured data or other objective functions. % TODO: Find information about objective functions from the paper
This leads to a point estimate of the likely intracellular fluxes, where the objective function is a proxy for the feasiblity of the fluxes.
However, a single point estimate may not faithfully represent the entire space of possible fluxes, for example if there are multiple modes of possible feasible fluxes.
An alternative is to characterise the entire space of vaiable fluxes, usually with some form of sampling.

% IDEA FROM PAPER - sampling is nice because the effect of any constraints can be easily examined - you split the samples into those that satisfy and don't satisfy the constraints

Sampling methods can be broadly split into Bayesian and non-Bayesian models.

Methods 1,2 {TODO: Research these} were developed to sample uniformly from the space of steady-state fluxes.
Saa {TODO: Citation} extended these methods to exclude loops from the flux space, which are known to violate the 2nd law of thermodynamics.
Alternatively, one can simply include thermodynamic constraints which addresses the problem of loops while excluding further fluxes that violate thermodynamics.
In this case, for all reacitons, the Gibbs' energy of reaction ($\dgr$) must be less than 0.
In practice, the $\dgr$ of reaction $i$ is calculated as the sum of the Gibbs energy of reaction under standard conditions ($\sdgr$) and the scaled logarithm of products over reactants (\ref{eq:dgr})
\begin{align}
    \dgr_i &= \sdgr_i + RT\ln \frac{P}{R} \label{eq:dgr}
\end{align}
where $R$ is the gas constant, $T$ is the temperature and $P$ and $R$ are the concentrations of the reaction's products and reactants respectively.
Because the Gibbs energy of a reaction is a state function, the $\dgr$ and $\sdgr$ of a reaction are equivalent to the Gibbs energy of formation ($\dgf$) and standard Gibbs energy of formation ($\sdgf$) respectively of all reaction substrates, when scaled by their stoichiometries.
That is,
\[
    \sdgr_i = s_i \cdot \sdgf_i
\]
where $s_i \in \mathbb{Z}^{m_i}$ are the stoichiometries and $\sdgf \in \mathbb{R}^{m_i}$ are the standard formations energies of metabolites $1,\dots,m_i$ involved in reaction $i$.

For a set of reactions $1,\dots,r$ ($\bdgr \in \mathbb{R^r}$), this can also be written in matrix form as
\begin{align}
        \bdgr &= \bsdgr + RT \cdot S^{T}\ln{\mathbf{c}} \label{eq:dgr_vec}\\
              &= RT \cdot S^T (\sdgf + RT \ln{\mathbf{c}}) \label{eq:dgr_sdgf_vec}
\end{align}
where $\mathbf{c} \in \mathbb{R}_{+}^m$ is the concentration of all metabolites $1,\dots,m$ in $S$.
Eq~\ref{eq:dgr_vec} shows the relation between the $\dgr$ and $\sdgr$ and Eq~\ref{eq:dgr_sdgf_vec} shows how the same can be calculated with the $\sdgf$.
This formulation also nicely highlights the fact that $\dgr$, $\bsdgr$ and $\ln{\mathbf{c}}$ are linearly related. % TODO: This is the wrong statement to be making
As these values are estimated, another benefit of Eq~\ref{eq:dgr_sdgf_vec} becomes clear, namely that all combinations of $\sdgf$ produce a system that is thermodynamically consistent.
An arbitrarily chosen set of $\sdgr$ does not satisfy this property (see~\cite{noor_2013_equilibrator} for a more thorough description of this effect).

% A more natural way to incorporate this probabilistic information would be to sample parameter sets accoring to their
% likelihood. Any optimisation after this point can be by selecting the best one but balancing the likelihood of it
% being present.

Both $\mathbf{c}$ and $\sdgf$ can not be known for certain and need to be estimated, which creates a degree of uncertainty about their true values.
To account for this uncertainty, some margin of error needs to be included to ensure that the true intracellular conditions can be represented.
TMFA {TODO: Etc} define this margin of error for $\sdgr$ and metabolite concentrations by constraining them individually to fall within the 95\% confidence interval.
% TODO: Whole paragraph on the multivariate approach? It's kinda relevant.
~\cite{multiTFA} and~\cite{PTA} improve upon this approach by constraining these values to a 95\% confidence interval of their joint probability, significantly tightening the bounds on possible values.
However, both of these methods discard useful probabistic information about these parameters, effectively treating all values within the confidence interval as equally likely.
~\cite{PTA} {TODO: The exact method} addresses this problem by sampling from the space of valid thermodynamic orthants, i.e. flux directions, with the probabilty of each orthant determined by the probability of $\sdgr$ and metabolite concentrations within it.
Unfortunately, this method still restricts the space of valid concentrations and $\sdgr$ to the 95\% confidence interval, although, to the authors' knowledge, this is not necessary.
Additionally, the hit-and-run sampling method described in~\cite{PTA} requires multiple linear programming evaluations per sample, to ensure that samples satisfy steady state (Eq~\ref{eq:steady_state}), which can inhibit computational performance.

% TODO: Talk about thermodynamics here. How hard and time-consuming it is to estimate. Problems with coverage of existing methods. What to do with unknown compounds?

In this work, I present an alternative approach to defining and sampling from the space of fluxes that satisfy both steady-state and thermodynamic constraints.
Generative Thermodynamic Flux Analysis (GTFA) is a method for sampling thermodynamically valid fluxes with respect to the probability that these fluxes match measured metabolite concentrations and standard formation energies.
We represent fluxes as a function of their $\dgr$ multiplied by a scaling factor and separate the network into a set of "fixed" and "free" fluxes.
This results in a parameter space where all combinations of parameters satisfy mass balance and thermodynamic constraints, as opposed to~\cite{PTA}, where large portions of the $\dgr$ parameter space are invalid and must be avoided.
Furthermore, GTFA allows the inclusion of metabolomics, proteomics and fluxomics data.
The final output of GTFA is then a set thermodynamically-valid fluxes, which respects the probability of each of these varied data sources.

% TODO: Maybe some examples? Slower explanation of how there is probabilistic information attached to all of these measurements?

\section*{Materials and methods}

\subsection{Model formulation}
GTFA unifies metabolomics, fluxomics and enzyme proteomics data of a metabolic network in a bayesian generative statistical model.
That is, all measurements are assumed to be subject to measurement error and have latent counterparts in the model that represent the true parameter values.
In the end, this defines a model that is able to generate measured data.
Structurally, fluxes and thermodynamic data are forced to satisfy both steady state~\ref{eq:steady_state} and thermodynamics.

This section will describe the statistical model used in GTFA.
As the model is quite complex, we will progressively increase complexity, starting with a simple model and progressively adding complexity until we reach the final model that was applied in the analyses.

% TODO: Give the models names??
% TODO: Define be as b x e
A naive first description of the model defines the fluxes as a function of the $\dgr$ scaled by enzyme concentrations and the scaling factor $b$, for a single condition.
Firstly, the likelihood function describes the densities of measured data for a given parameter configuration.
The fluxes $\bv$ are a function of the free parameters of the system $\bsdgf$, $\be$, $\bb$ and $\bc$:
\begin{align}
    \bv &= \bdgr \odot \be \odot \bb \\
        &= S^T (\bsdgr + RT\ln{\bc}) \odot \be \odot \bb \\
\end{align}
.
Measurement densities are described as follows:
% TODO: The notation here is a little lax. How can we do it nicely showing that this is conditional on the params?
\begin{align*}
    \mathbf{y_{e}} ~ LN(\ln{\be}, \Sigma_{me}) \\
    \mathbf{y_{c}} ~ LN(\ln{\bc}, \Sigma_{mc}) \\
    \mathbf{y_{v}} ~ N(\bv, \Sigma_{mv}) \\
\end{align*}
where $\Sigma_{me}$, $\Sigma_{mc}$ and $\Sigma{mv}$ are the covariance matrices of the respective measurement errors.
As the measurement errors are assumed to be independent, these covariance matrices are diagonal and thus uniquely defined by their traces $\mathbf{\sigma_{be}}$, $\mathbf{\sigma_{bc}}$, $\mathbf{\sigma_{bv}}$.

The priors are defined as follows:
\begin{align*}
    \be ~ LN(\mathbf{\mu_e}, \Sigma_e) \\
    \bb ~ LN(\mathbf{\mu_b}, \Sigma_b) \\
    \bc ~ LN(\mathbf{\mu_c}, \Sigma_c) \\
    \bv ~ N(\mathbf{\mu_v}, \Sigma_v) \\
    \bsdgf ~ N(\mathbf{\mu_{f}}, \Sigma_{f}) \\
\end{align*}
where $\mathbf{\mu_e}$, $\mathbf{\mu_c}$, $\mathbf{\mu_v}$, $\mathbf{\mu_b}$ and $\mathbf{\mu_f}$ are the means of the prior distributions and $\Sigma_e$,$\Sigma_b$, $\Sigma_c$,$\Sigma_v$, $\Sigma_f$ are the covariance matrices of the priors distributions.
The covariance matrices $\Sigma_e$, $\Sigma_b$, $\Sigma_c$ and $\Sigma_v$ assume independence between the individual entries with traces of $\mathbf{\sigma_e}$, $\mathbf{\sigma_b}$, $\mathbf{\sigma_c}$ and $\mathbf{\sigma_v}$.
For priors on the enzyme and metabolite concentrations we apply the same method as~\cite{PTA} and fit a normal distribution to all conditions in the dataset. %TODO: Actually do this and put the values here
% TODO: Do we have priors on v? If so, how? We could also use the fluxomics data
Defining a prior for $b$ is more difficult, as there is little prior information on such a quantity and it is expected to vary greatly across reactions.
% TODO: Mention Lars' thing here?
We somewhat arbitrarily used a lognormal distribution with a mean of $x$ and a sd of $y$. %TODO, what are the actual values?
% TOdo: Refer to actual analysis here.
Defining a prior for $\bsdgf$ was more complex process as it relied on estimates provided by equilibrator~\cite{noor_2013_equilibrator}.
This is described more thoroughly in Section~\ref{sec:sdgf_estimation}.
\begin{align}
    \bv &= \bdgr \odot \be \odot \bb \\
        &= S^T (\bsdgr + RT\ln{\bc}) \odot \be \odot \bb \\
\end{align}
.
Measurement densities are described as follows:
% TODO: The notation here is a little lax. How can we do it nicely showing that this is conditional on the params?
\begin{align*}
    \mathbf{y_{e}} ~ LN(\ln{\be}, \Sigma_{me}) \\
    \mathbf{y_{c}} ~ LN(\ln{\bc}, \Sigma_{mc}) \\
    \mathbf{y_{v}} ~ N(\bv, \Sigma_{mv}) \\
\end{align*}
where $\Sigma_{me}$, $\Sigma_{mc}$ and $\Sigma{mv}$ are the covariance matrices of the respective measurement errors.
As the measurement errors are assumed to be independent, these covariance matrices are diagonal and thus uniquely defined by their traces $\mathbf{\sigma_{be}}$, $\mathbf{\sigma_{bc}}$, $\mathbf{\sigma_{bv}}$.

The priors are defined as follows:
\begin{align*}
    \be ~ LN(\mathbf{\mu_e}, \Sigma_e) \\
    \bb ~ LN(\mathbf{\mu_b}, \Sigma_b) \\
    \bc ~ LN(\mathbf{\mu_c}, \Sigma_c) \\
    \bsdgf ~ N(\mathbf{\mu_{f}}, \Sigma_{f}) \\
\end{align*}

% TODO: Other kinds of excluded reactions might need to be mentioned here
% TODO: Have numbered models?
\subsubsection{Exchange reactions}
The first adaption to the model is the inclusion of exchange reactions.
Material needs to enter and exit the model, which is represented by one-sided "exchange" reactions that bring individual metabolites through the system boundaries.
These reactions need to satisfy mass balance, but not thermodynamic constraints.
In practice, this requires the flux vector $\bv$ and the stoichiometric matrix $S$ each be separated into two parts.
We represent the first $1,\dots,p$ columns of $S$ and elements of $\bv$ as $S_\Pi$ and $\bv_\Pi$ as those corresponding to the $p$ exchange reactions in the network.
This leaves the remaining $1,\dots,g$ columns of $S$ and elements of $\bv$ as $S_\Gamma$ and $\bv_{Gamma}$ to represent the $g$ internal reactions in the network with $g+p=r$.
Because there is no meaningful notion of thermodynamics associated with these reactions, they cannot be represented in the same manner as the previous model.
Accordingly, the vectors $\be$ and $\bb$ now only have $g$ elements and $\bdgr_{\Gamma} = S\cdot \bdgf$ refers to the Gibbs energy of internal reactions.
The model then defines internal fluxes as before
\begin{align}
    \bv_\Gamma &= \bdgr_{\Gamma} \odot \be \odot \bb \\
        &= S_{\Gamma}^T (\bsdgf + RT\ln{\bc}) \odot \be \odot \bb \\
\end{align}
with a normal prior on the exchange fluxes, which are now free parameters
\begin{align*}
    \bv_{\Pi} ~ N(\mathbf{\mu_{v_{\Pi}}}, \Sigma_{v_{\Pi}}) \\
\end{align*}
This formulation highlights the fact that, with exchange fluxes allowed to vary freely and internal fluxes dependent upon other parameters, there is no guarantee that the steady-state condition holds.
In fact, it is impossible for the constraints to hold, which will be addressed in the next section.

\subsubsection{Parameter fixing}
% TODO: Include here what to do with dependent rows of the s matrix as well. What did I do again?
Equation~\ref{eq:steady_state} effectively removes $m$ degrees of freedom from the system by requiring mass balance of metabolites.
To ensure that all parameter configurations are valid, these degrees of freedom need to be removed from the system.
Often a non-linear solver will be applied to determine a set of fixed parameters that satisfy the given constraints from the parameters that are free to vary. %TODO: Reference for this? Can I reference MAUD?
Luckily, because of the aforementioned linear dependence between $v$, $\bsdgf$ and $\ln{\bc}$, the fixed parameters can easily be determined as a linear combination of free parameters.
\\
If we rearrange these linear relations into a single matrix, the process of solving for fixed parameters is greatly simplified.
For this purpose we require a number of definitions:
\begin{align}
    S_v = \begin{bmatrix}
             I^p & \mathbf{0} \\
             \mathbf{0} & S_{\Gamma}^T \cdot \bbe\\
          \end{bmatrix} \\
    x = \begin{bmatrix}
            \bv_{\Pi} \\
            \bsdgf + RT\ln{\bc}\\ % TODO: Should we just notate these as formation energies?
        \end{bmatrix} \\
    S_m = S \cdot S_v\\
\end{align}
with $x \in \mathbb{R}^{m+p}$ being a vector of exchange reactions and formation energies, $S_v \in \mathbb{R}^{r \times m+p}$
 the matrix that transforms this vector to a vector of fluxes and $S_m \in \mathbb{R}^{m \times m+p}$ the matrix that
 transforms $x$ to the change in metabolite concentrations $d\bc/dt$, which is 0 under steady state.

%TODO: Think about the exact sizes of everything here. Will have something to do with the rank of the matrix
With these definitions in place, $x$ can be split into two parts $x^{fi} \in \mathbb{R}^{ra(S_m)}$ and $x^{fr}\in \mathbb{R}^{m+p-ra(S_m)}$ that correspond to the basic and free variables of $S_m$.
% TODO: We need a reference for the above
% TODO: Should we be more precise here? It's probably not necessary
The values of the free varaibles can thus be determined given a set of proposed free variables
\begin{align}
    0 &= S_m\cdot x\\
    -S_m^{fr}\cdot x^{fr} &= S_m^{fi}\cdot  x^{fi}\\
    x^{fi} &= S_m^{fi}^{-1} \cdot -(S_m^{fr}\cdot x^{fr})\\
\end{align}
where $S_m^{fi}$ and $S_m^{fr}$ are matrices composed of the pivot and non-pivot columns of $S_m$ respectively.
In practice, $x^{fi}$ is solved for instead of using the $S_m^{fi}^{-1}$ due to numerical issues.
This separation into fixed and free fluxes can be extended to the components of $x$, with $\bc^{fi}$, $\bc^{fr}$, $\bv_\Pi^{fi}$ and $\bv_\Pi^{fr}$ corresponding to the fixed and free metabolite concentrations and exchange fluxes respectively.
In concrete terms, this means that the model proposes a set of free fluxes and this allows the determination of the rest of the fluxes such that they satisfy steady-state constraints.
This gives a model with the fixed variables of $x$ determined by free varaiables:
\begin{align}
    v &= S_m\cdot x\\
\end{align}
This requires modified priors
\begin{align}
    \bv_\Pi^{fr} ~ N(\mathbf{\mu_{v_{\Pi}}}, \Sigma_{v_{\Pi}}) \\
    \bc^{fi} ~ LN(\mathbf{\mu_c}, \Sigma_c) \\
    \bc^{fr} ~ LN(\mathbf{\mu_c}, \Sigma_c) \\
\end{align}
for the model.
The prior for $\bc$ is interesting in that $\bc^{fi}$ are not true parameters of the model.
This means that the prior on $\bc^{fr}$ is in fact an (inadmissable) combination of the direct prior and a transfomration of the transformed second prior on $\bc^{fi}$.
This is further complicated by the fact that, as this is a transformation of paramters, a jacobian transformation should be included to account for the "stretching" of the parameter space by this transformation.
In Section \ref{sec:x_prior} we show that this had no effect in practice and simplcity of formulation was preferred over any theoretical concerns.
%TODO: The above prior is not "inadmissible" - there is another word for it
% TODO: Should I also have a prior on the fixed exchange fluxes?
% TODO: Make a table of the full models with likelihood and prior
% TODO: Redundant rows (Does this change the final equation if the inverse is used?)
% TODO: Mention that S transforms fluxes to changes in met concentrations at the top
% TODO: Which law of thermodynamics?
%TODO: Is dgr_m a nice symbol for the membrane potential? What do others use?
%TODO: Replace the x part with \bdgf

\subsubsection{Membrane potentials}
% TODO: HOW ARE MEMBRANE POTENTIALS CALCULATED?
The final aspect that needs to be included into the model is the addition of membrane potentials for intercompartmental reactions.
This manifests itself as a single value that must be added to the $dgr$ of each reaction.
This requires $S_v$ and $x$ to be slightly refactored.
$S_v$ now includes an extra first column that contains the membrane potentials $\dgr_m \in \mathbb{R}^p$ and $x$ a first entry with the value $1$.
This gives
\[
    S_v = \begin{bmatrix}
              \mathbf{0} & \mathbf{I^p} & \mathbf{0} \\
              \dgr_m & \mathbf{0} & S_{\Gamma}^T\\
          \end{bmatrix} \\
    x = \begin{bmatrix}
            1 \\
            \bv_\Pi \\
            \bsdgf + RT\ln{\bc} \\
        \end{bmatrix}
\]
leaving all other values unchanged.
\subsection{Estimation of $\sdgf$} \label{sec:sdgf_estimation}

% TODO: Talk about datasets
An important part of the modelling process is determining priors for the $\sdgf$ of the metabolites in the metabolic network.
The standard method for estimating $\sdgf$ and $\sdgr$ has been to use Equilibrator \cite{noor_2013_equilibrator}, which combines the reactant and group contribution methods.
The reactant contribution (rc) method performs a linear regression to determine the least squares estimates for standard formation energies of metabolites in the training set
\[
    \label{eq:rc}
    \mathbf{\bsdgr_x} = S_x^T\bsdgf_x + \eps_{rc}
\]
where $\mathbf{\bsdgr_x}$ is the measured $\bsdgr$ of the training set, $S_x$ is the stoichiometric matrix of the training data and $\eps_{rc} ~ N(0, \sigma_{rc})$ accounts for measurement error. % TODO: And what else?
The estimates for $\bsdgf_x$ and $\bsdgr_x$ are then obtained with the standard normal equations
\begin{align}
    E[\hat{\bsdgf_x}] = (S_x^T)^+ \mathbf{y}\\
    Var(\hat{\bsdgf_x}) = (S_x S_x^T)^+ \\
    E[\hat{y}] = \mathbf{S_x} \cdot E[\hat{\bsdgf}] \\
    Var(\hat{\bsdgr}) = \mathbf{S_x} Var(\hat{\bsdgf}) \mathbf{S_x.T} + \hat{\Sigma_{rc}} \\
    \hat{\sigma_{rc}} = s^2 = \frac{\Sum_{i=1}^{N} (E[\hat{\sdgf_i}] - \sdgf_i)^2}{N-\text{rank}(S)}
\end{align}
where $\Sigma_{rc}$ is a diagonal matrix whose elements are $\sigma_{rc}$, $s^2$ is the standard error and $\sdgf_i$ is reaction $i$ of the training dataset.
This is clearly a lienar regression with no intercept, which matches the intuition that a reaction with no change should have a $\sdgr$ of 0.
% TODO: Maybe I need a variable for the rank of S
% TODO: Define domains of the dataset so that variance estimates make sense and correct the N above if it should be something else.
% TODO: Change the x vector to a different name.
This predominantly relies on data on the equilibrium constants of biochemical reactions, which sometimes need to be corrected to standard conditions. %TODO: How?
The linear regression accounts for the (relatively small) measurement error in these experiments, as well as small errors in the correction to standard condition. % TODO: Check what they actually said
While being highly accurate, the reactant contribution method only determines the formation energies of metabolites in the training dataset and thus has relatively low coverage.
Furthermore, conserved metabolites (TODO: Check this name), i.e. linearly dependant rows of $S$, cannot be fully determined by the method.
As seen in Equation \ref{eq:rc}, this is the reason that the Moore-Penrose pseudoinverse, instead of the regular matrix inverse, is required for the estimation of the linear regression coefficients.
A consequence of this, is that the rc method only correctly estimates the component of these dependent rows that exists in the range of the linear transformation described by $S$.

For example, the cofactors ATP and ADP are ubiquitous in the training data, but only as a coupled pair.
This means that, while the rc method can accurately estimate their relative formation energies, it is impossible to estimate their absolute formation energies.
Any changes to their absolute formation energies that maintain the same relative energies cause no change to the $\sdgr$ of the reactions in the dataset, leaving a degree of freedom.
Formally, this means that the vector $[1,-1]$ in $[\sdgf_{\text{ATP}}, \sdgf_{\text{ADP}}]$ is part of the range of $S$, while $[1, 1]$ is in the null space of $S$.
A side-effect of the Moore-Penrose pseudoinverse is that these $\sdgf$ estimates have a mean of 0, that is, if the optimal $\sdgf_{\text{ATP}} - \sdgf_{\text{ADP}}$ is $10kJ/Mol$, the two estimates would be $-5$ and $5$.

On the other hand, the group contribution (gc) method trades some accuracy for a much greater coverage.
The assumption behind the gc method is that metabolites can be broken down into counts of their constituent chemical moieties, also known as groups, as shown in Figure \ref{fig:gc}.
A linear regression is then able to estimate the $\sdgr$ with the change in groups over the reaction.
\begin{align}
    \bsdgr = S^T G \bsdgg + \eps_{gc}\\ %TODO: eps Should be bold?
\end{align}
where $G \in \mathbb{R}_(+)^{m \times g}$ is a matrix containing the counts of chemical moieties $1,...,g$ and $\bsdgg \in \mathbb{R}^{g}$ is the formation energy of each chemcial moiety.
That is, $G \bsdgg$ is a set of estimates for $\bsdgf$, which are then transformed by $\S^T$ to $\bsdgr$ estimates.
Because metabolites are broken down into a common set of groups, this allows the results of the regression to be applied to reactions that contain metabolites that were not present in the original training set.
Unfortunately, this assumption is not completely valid with other factors (for example proximity or more complex conformational factors) also playing a role in a compound's foramtion energy.
Consequently, the $\eps_{gc}$ includes not only the measurement error, but also the sum of these other errors as well, making the predictions less accurate.

\cite{noor_2013_equilibrator} elegantly leverages the linear algebra underlying linear regression to unify the two methods, combining the accurate estimates of reactant contribution with the increased coverage of group contribution.
Assembling a set of $\sdgr$ values from different sources in a single network can lead to violations in the second law of thermodynamics, as shown in Figure \ref{fig:cc}.
An alternative is to estimate the $\sdgf$ of all metabolites within a reaction, leaving the $\sdgr$ as the sum of all metabolites multiplied by their stoichiometric coefficients which, by definition, ensures conservation of energy. % TODO: Do I have an equation already?
This suggests a clear path for combining the two methods: use reactant contribution estimates for $\bsdgf$, and use group contribution estimates for the rest.
This simple approach is unfortunately complicated by the not-insiginificant set of degenerate metabolites in the training set. % TODO: Is this the right term? They called them conserved something-or-other
Cofactor pairs cannot be fully determined by the $rc$ method, but their relative formation energies can.
If their formation energies were replaced with those of the $gc$ method, this accurate information would be lost.
The solution is to combine estimates from both by projecting the $rc$ $\bsdgf$ estimate onto the range of $S$ with the projection matrix $\projr$ and the $gc$ estimate onto the null space of $S^T$ with the projection matrix $\projn$.
Here, $\mathit{R}$ and $\mathit{N}$ define the functions that return the range and null space of a matrix respectively.
In the case of degenerate cofactor pairs, $\mathit{N}(S^T)$ defines precisely their absolute formation energies, while the range of $S$ defines their relative formation energies.
When compounds do not participate at all in any training set reactions, they are by definition in the null space of $S^T$ and are estimated entirely with the $gc$ method.
Conversely, if they are fully determined by reactant contribution, they are not present at all in the null space of $S^T$, and are estimated wholly with the $rc$ method.
Estimates based on the equliibrator method can thus be described by the following equation:
\[
    \hat{\bsdgr_x} = \mathbf{X^T}(\projr\bsdgf_{rc} + \projn\bsdgf_{gc})\\
\]
where $\hat{\bsdgr_x}$ are the estimates for the matrix $X$ containing the stoichiometry of the reactions being estimated. % TODO: Didn't I define this above already?
Further detail can be found in \cite{noor_2013_equilibrator}, with a very thorough explanation in the supplementary material of \cite{equilibrator_3_beber_noor}.

Unfortunately, all estimate uncertainty has been defined in terms of the confidence interval.
It appears that this is not unique to Equilibrator, but has been the only measure of uncertainty up to this point.
This is problematic, because we are modelling ranges for the true $\bsdgr$ or $\bsdgf$ of the compounds and reactions, while the confidence interval only relates to the estimate.
A more appropriate measure of uncertainty would in fact be the prediction interval, which includes the influence of random error $\eps$.
% TODO: Notation? Show the prediction interval here
This has a number of effects on the application of these variance estimates.
Firstly and most obviously, the prediction interval gives a higher marginal varaiance for the $\bsdgr$.
This makes intuitive sense, because with increasing data, the uncertainty about the coefficients drops, even if the linear relationship is weak.
This leads to a confidence interval which converges to the true parameter value, leaving no uncertainty, even though a weak linear relationship reveals little about the true value of the predicted value.
This suggests that models up to this point have been to strict in their bounds on $\sdgr$ of reactions by confining them to their 95\% confidence interval.
This is especially true of $\sdgr$ estimates using the group contribution method, which has a large residual error.
% TODO Include the methods for estimation above
% TODO: Add the equations here and maybe include in the above paragraph
Secondly, the inclusion of the independent errors in the prediciton interval guarantees a full rank covariance matrix for the estimates.
This matches one's intuition for these estimates, for example in the case of metabolites that have identical group vectors.
The covariance matrix of the confidence interval of these metabolites is necessarily degenerate because their estimates will always be identical.
Any sampling strategy based on this matrix, for example the Cholesky decomposition applied in \cite{multiTFA}, \cite{gollub_2021_prob_sampling} and \cite{equilibrator_3_beber_noor},
 reduces the dimensionality of the normal distribution being sampled from and does not allow these estimates to vary from each other, effectively forcing them to be identical.
Intuition suggests that, while their estimates will be identical, these values should be allowed to vary from each other.
This suggests an alternative approach for sampling from the $\bsdgr$ estimates from Equilibrator:
\[
    PI(\bsdgr_x^{cc}) = CI(\bsdgr^{cc}) + \mathbf{s^2}^{cc}
\]
where $\mathbf{s^2}^{cc}$ is a vector of the appropriate sample variances of each prediction.
That is $\mathbf{s^2}^{cc} = [\mathbf{s^2}^{rc}, \mathbf{s^2}^{gc}]$, where all reactions purely estimated with $rc$ are placed before those with $gc$ components.
% TODO: This DEFINITELY needs to be more thought through
% TODO: Should we still be using the projections?
% Think a little more about how this could be done.


% TODO: Each new variable should be given a domain

% TODO: Notation m_x and m_s?
% TODO: We're not really looking at the "interval" here per se. Is there another term to use?
Understanding the prediction and confidence intervals of $\dgf$ is even more complicated because there little data on this property.
That is, almost all data on formation energies is inferred from known reaction equilibria and their corresponding $\sdgr$.
In the case of the group contribution method, it is clear that $S^T\hat{\bsdgg)}$ does not approach the true value of $\bsdgf$.
$\hat{\bsdgg}$ converges to the true value, but when transformed to an estimate of $\sdgf$, as in \cite{noor_2013_equilibrator}, these estimates converge to values inconsistent with the second law of thermodynamics.

% TODO: Think more about about a counterexample here. Basically, two reactions that have the same group change. The formation energies converge to a value that isn't correct.
% THE ARGUMENT - given some true reaction values, the values that ST G converges to don't satisfy the 2nd law of thermo.

% TODO: We need a good toy example here.
% TODO: Perfect dgr from dgf as well as dgr = ST G beta_g + eps
% TODO: Does this show that G beta_g does not coverge to the real values?

% TODO: This argument needs to be clearer

We propose a simple method to address this problem, by modelling the formation energies as reactions.
In the case of metabolites present in the training set, this is guaranteed to be a single covariate.
From this perspective all of the relevant errors are able to be estimated and this solves the problem of a rank-degenerate covariance matrix.
While this is not a theoretically sound solution, we believe that this is a step in the right direction and solves all of the above-named problems with current variance estimates.


% TODO: Mention that we're moving from estimates of sdgr to sdgf and that this might be a problem

% TODO: Example daigram showing cc method with mix and match

% TODO: Figure out how to do author citations
% TODO: The group contribution method is also performed on the reactions! The error is a reaction-specific error.

% TODO Diagram of the gc method with a compound being broken down into its components.
% TODO: Get specifics of RMSE of gc and rc from equilibrator
% TODO: Did the first equilibrator offer the covaraince or was it only later?

\subsection{Datasets}

GTFA was applied to a number of datasets to demonstrate the sensibility of posterior parameter estimates.
The first example was borrowed from \cite{PTA} and consists of 4 metabolites, 5 internal reactions and 2 exchange reactions.
This example is constructed to demonstrate the interaction between $\bv$ and $\bdgr$, with large infeasible regions in $\bdgr$ that correspond to infeasible flux directions.
In fact, the network only has four possible direcitonality configurations, shown in Figure \ref{fig:toy_configs}.
This dataset was augmented by a set of enzyme, flux and concentration parameters generated using the equations of Model \ref{mod:final}.
This allows the generation of data consistent with the GTFA sturcutral assumptions which was then used for a simulation study, ensuring that true parameters could be recovered.

GTFA was also applied to a second simulated dataset based on the methionine cycle.
In this case, the data was simualated from a more complex kinetic model of metabolism consisting of $x$ reactions and $y$ metabolites. % TODO: Fix



\subsubsection{Gollub dataset}
% Need more info here

% Results and Discussion can be combined.
\section*{Results}
% TOY DATA
When examining the $\bdgr$ of the samples, we see the same pattern as in \cite{PTA}, with samples only occurring in regions that allow feasible fluxes.
Technically, it is feasible that samples in these regions occur, but only in combination with $b$ or $e$ values that give a corresponding flux $v$ below the error threshold of the linear equation solver.

\subsection{Prior on elements of $x$}
\label{sec:x_prior}

\subsection{The effect of change of variables}

\subsection{Speed comparison}

\subsection{}

\section*{Discussion}

% Order of x (and S) in assigning fixed and free varaiables.

% The S_m matrix changes with every proposal. It's possible to use symbolic algebra if we wanted to speed it up.



% GTFA in the context of a pre-processing step before kinetic modelling
% We are trying to unify many sources of information.
% If we represent the results with (eg) a normal distribution, why not use variational inferece?
%

\subsection{Future work}

This work represents a first description of a novel approach to Bayesian modelling of metabolism.
There are therefore many avenues to explore and questions to answer in the future.

% Parameter concerns
First of all, the effect of this formulation on parameter estimation has yet to be fully explored.
Of concern is the possible high correlation between $\bdgr$, $\bb$ and $\be$, all of which are drivers for increased flux.
If the $\dgr$ of a reaction is not well defined, for example for a reaction whose substrates' $\sdgf$ cannot be estimated, the $\dgr$ estimate may be higher than the true estimate, simply to drive the reaction forward.
This equation is by definition unphysical, the lack of relationship between $|\dgr|$ and $v$ is universally known and accepted.
An alternative may be to have the effect of $\dgr$ on the flux saturate.
However, this unfortunately breaks the useful linear relationship between $\bv$ and $\log{\bc}$.
Another problem is the flux scaling parameter $b$, which is difficult to characterise with a prior.
One option is to refer the the literature of $k_{app}$ {TODO: Is this correct?}.
% TODO: Write more about the k apparent here.

The speed of GTFA make it applicable for a number of exciting applications.
A primary target could be the estimation of the $\sdgf$ of compounds that cannot be estimated by current methods, such as equilibrator.
As the number of model conditions increases, our estimates of the true shared $\sdgf$ parameters also improves, including compounds without prior information.
This could be invaluable for further modelling, where tighter bounds on the thermodynamic properties of compounds allow for more accurate results.
Additionally, GTFA could be very interesting as a "first pass" method for generating priors for more complex kinetic models.
% TODO: Possible applications for the approach as it is
% Because it is fast we could use it simultaneously on many conditions at once -> Narrow down on thermodynamic params

Extensions to the model that incorporate more measureable data would allow for tighter fits and more accurate parameter estimates.
One key element would be the inclusion of direct fluxomics data into the model.
Currently, fluxomics data is provided in the form of measured fluxes through reactions.
These can either be inferred through Eueler's method {TODO: Check this}, that is, by inferring exchange fluxes by measuring the changes in extracellular metabolite concentrations or by tracking the movement of labelled atoms through the system.
A constraint-based method called Metabolic Flux Analysis (MFA) is typically applied {TODO: Citation} to find the most likely flux distribtuion given this set of labels, but recently Bayesian alternatives have also been developed {TODO: Citation}.
Importantly, these concentrations of labelled metabolites provide information not only on the likely fluxes, but also about reaction thermodynamics.
While the net flux passing through an enzyme is largely independent of $\dgr$, the forward and reverse fluxes decidedly are.
When a reaction is at equilibrium, which is rarely the case in biology, the rates of the forward and reverse reactions are equal, but not 0.
This is an important consideration when the reactant and product pools contain differently-labelled atoms, a process called mixing. %TODO: is it?
This suggests that the degree of mixing between the reactants and products of a reaction, when taken in the context of the overall fluxes, can reveal some information about the thermodynamics of this reaction under the measured conditions.
The integration of reaction thermodynamics into Bayesian MFA could be a very interesting avenue of research into the future, providing a further source of systems-wide information into reaction thermodynamics.

An alternative for the estimation of $\sdgf$ of unknown compounds (and thus the $\sdgr$ of reactions containing those compounds) is to discern the relationship between compound structure and formation energy.
\citet{quantum} attempted to determine the formation energies from first principles using quantum chemistry, but unfortunately did not continue this line of research. %TODO Read the paper
Alternatively, one can leverage machine learning methods to discern the functional relationship from existing data.
Unfortunately, due to the time-consuming process of gathering formation energy and reaction equilibrium data, there is not a large amount of data available on which to train these models.
This is in contrast to many other forms of compound-level data, for example 3D structure, toxicity etc. % TODO:
One option is to apply the technique of transfer learning, training a neural network to perform a task with more abundant data and then adapting this semantically-rich feature representation to the new data-poor task.
\cite{3dinfomax} applied transfer learning to a range of molecular regression and classification tasks, by applying a feature representation optimised for predicing the compounds' 3D structures.

An alternative, is to use machine learning or statistical models to directly predict latent parameters of Bayesian models.
In this case, this would involve using ML models to predict the $\sdgf$ of the compounds in the network, with the likelihood of these parameters representing the loss function of the ML model's parameters.
The parameters of the machine learning model could then be optimised either within the Bayesian model itself (with MCMC) or with traditional methods (such as gradient descent) with this new loss function.
This leverages the fact that more abundant -omics data and an accurate model structure provide information about likely thermodynamic parameters, even when these parameters cannot be measured directly.
For example, by integrating the linear regression of the group contribution method (Eq. \ref{eq:gc}) into GTFA, it would be possible to estimate the formation energy for groups that are not present in the original training dataset, expanding the coverage of the method to compounds that include that group.
Even a small amount of data could considerably reduce the uncertainty in these groups, which are currently modeled as normal with mean 0 and standard deviation 10,000.
Alternatively, more complex graph neural networks could be applied to the same task, now with much larger datasets available.
Of course, the computational complexity of embedding the data in Bayesian models could quickly become and issue, especially in this case where many conditions, each with their own model parameters, would be required.
Nevertheless, making use of these ever-growing -omics datasets could be the only option if more reaction equilibrium data does not become available.





% TODO: Talk about quantum chemistry paper


% Could gtfa work without fluxomics data - i.e. to predict fluxes?
% Inclusion of fluxomics data directly - forward and reverse data
% Inclusion of predictive model into bayesian model? Can you develop a machine learning method of learning latent paramters?
%
% Proper formulation of the prior for x

% Membrane potentials should be included into the model.

\section*{Conclusion}

% TODO: Remove this eventually
\bibliography{bibliography}

\section*{Supporting information}

% Include only the SI item label in the paragraph heading. Use the \nameref{label} command to cite SI items in the text.
%\paragraph*{S1 Fig.}
%\label{S1_Fig}
%{\bf Bold the title sentence.} Add descriptive text after the title of the item (optional).

\section*{Acknowledgments}

\nolinenumbers

% Either type in your references using
% \begin{thebibliography}{}
% \bibitem{}
% Text
% \end{thebibliography}
%
% or
%
% Compile your BiBTeX database using our plos2015.bst
% style file and paste the contents of your .bbl file
% here. See http://journals.plos.org/plosone/s/latex for
% step-by-step instructions.



\end{document}

%------------------------------------------------------------------------------------------------------------------------
% EXAMPLES
% Place figure captions after the first paragraph in which they are cited.
%\begin{figure}[!h]
%\caption{{\bf Bold the figure title.}
%Figure caption text here, please use this space for the figure panel descriptions instead of using subfigure commands. A: Lorem ipsum dolor sit amet. B: Consectetur adipiscing elit.}
%\label{fig1}
%\end{figure}
%% For figure citations, please use "Fig" instead of "Figure".
%
%
%% Place tables after the first paragraph in which they are cited.
%\begin{table}[!ht]
%\begin{adjustwidth}{-2.25in}{0in} % Comment out/remove adjustwidth environment if table fits in text column.
%\centering
%\caption{
%{\bf Table caption Nulla mi mi, venenatis sed ipsum varius, volutpat euismod diam.}}
%\begin{tabular}{|l+l|l|l|l|l|l|l|}
%\hline
%\multicolumn{4}{|l|}{\bf Heading1} & \multicolumn{4}{|l|}{\bf Heading2}\\ \thickhline
%$cell1 row1$ & cell2 row 1 & cell3 row 1 & cell4 row 1 & cell5 row 1 & cell6 row 1 & cell7 row 1 & cell8 row 1\\ \hline
%$cell1 row2$ & cell2 row 2 & cell3 row 2 & cell4 row 2 & cell5 row 2 & cell6 row 2 & cell7 row 2 & cell8 row 2\\ \hline
%$cell1 row3$ & cell2 row 3 & cell3 row 3 & cell4 row 3 & cell5 row 3 & cell6 row 3 & cell7 row 3 & cell8 row 3\\ \hline
%\end{tabular}
%\begin{flushleft} Table notes Phasellus venenatis, tortor nec vestibulum mattis, massa tortor interdum felis, nec pellentesque metus tortor nec nisl. Ut ornare mauris tellus, vel dapibus arcu suscipit sed.
%\end{flushleft}
%\label{table1}
%\end{adjustwidth}
%\end{table}